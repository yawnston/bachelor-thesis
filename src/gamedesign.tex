\chapter{Game Design}
\label{chap:gamedesign}

This chapter's purpose is to discuss the considerations which went into designing
the~game's rules, and to describe said rules in detail.

The game is set on Mars with futuristic themes. In the game universe, Mars
is only just starting to be settled by humans, and there was a precious mineral
found under the surface - Omnium. This triggered a rush of colonists, who are
eager to make some profit. In the game, they compete for resources, and they all
want to build the largest colony, because the person with the largest colony
can extract the most Omnium and get rich.

\section{High Level Design}

\emph{Colonizers} has a few design decisions which are inherently set in stone by the premise
of this thesis:
\begin{itemize}
    \item The game must have more than two players
    \item The game must feature hidden information
    \item The game must support AI
\end{itemize}

AI support is only tangentially related to the design of the game rules, therefore
we will not discuss it at length in this chapter. We will focus on the other two
requirements.

\emph{Colonizers} is a four-player game. Four was chosen as a sweet spot for complexity,
since with five players, the game would start to get prohibitively expensive to compute.
It could be argued that three would accomplish the same goal, but four makes more sense
with respect to having enough design space. Four players is also a very common
player count for board games.

Hidden information is somewhat more tricky to get right. It can be implemented in many
ways, but even the simplest inclusions make the game much more tricky to process
with AI. There are two elements of hidden information in \emph{Colonizers}:
\begin{itemize}
    \item Players' hands and the Deck
    \item Players' colonists
\end{itemize}
These elements will be explained in more detail in the following section. It is worth
noting however that it is possible to make information-gathering plays, even though
only in very limited ways and in rare circumstances. Information-gathering plays
are not a large design focus of \emph{Colonizers}.

The game also features interaction between players --- both malicious and cooperative.
This naturally means that it is possible for multiple players to cooperate in order
to gain an advantage, or to conspire against another player in order to damage
that player's chances of winning. Many traditional AI algorithms are not capable
of cooperation or conspiracy, which provides room for specialized AIs to shine.

\emph{Colonizers} is technically not finite, since there exists a~strategy which,
if employed by all players, will lead to the game never ending. The game also
has potentially infinite states. In practice this
is incredibly unlikely, since the strategy involves players intentionally
passing up plays which would move them closer to the victory condition.

\section{Game Rules}
\label{chap:gamerules}

The game is played in rounds, which consist of turns. Turns then consist of phases.
The four players start the game in a given order, and they always take turns
in this order for the entire game.

Each player has a colony where they can build modules.
Each module has a point value, and the goal of the game is to build the most valuable colony.
When any player builds eight modules in their colony, the game will end at the end
of that round, after the remaining players have taken their turns.
When the game ends, the values of all modules in each player's colony
are added up, and the resulting value is that player's final score. Players can also
get a bonus to their score if they reached eight modules in their colony before
the game ended --- the first player to build eight modules gets four bonus points,
and other players to build eight modules get two bonus points each.
\footnote{As an example, assume that all four players have seven modules in their colony.
When player 1 takes their turn, they do not build a module, ending the turn with seven modules.
Player 2 then builds a module on their turn, taking them to eight modules in their colony.
This triggers the game end condition, but the game is not over until all players have taken
their turn this round. Since player 2 was the first to reach eight modules, they get four
bonus points. Player 3 then also builds a module, taking them to eight. Since player 3
was not the first to build eight modules, they get two bonus points. Player 4 then does
not build a module, ending the game with seven modules and zero bonus points. When player 4
finishes their turn, the game ends.}
The final ranking of the players when the game ends is determined by points --- players
with more points rank higher. If multiple players are tied in points, the player whose
position according to the player order is earlier ranks higher.
More information about modules and ways to interact with them follow in subsequent subsections.

There is also a rare game end condition, which is triggered by players attempting to
draw from an empty deck
\footnote{The deck is discussed in more detail in \Cref{de:modules}}.
This immediately ends the game with a draw, giving all
players zero points and a rank of zero.

\subsection{Colonist Pick}

At the start of each round, players take turns picking colonists. A colonist is~a~character
with special powers, and the player controls a given colonist for one turn. A player's
colonist is hidden from the other players.
There are six colonists in the game (see \Cref{gamedesign:colonists} for a~list
of available colonists). At the start of the colonist picks, a random colonist is
secretly removed for play for the rest of the round. Then, players take turns
picking from the remaining colonists one by one. This means that after the last
player picks, there is one colonist left over. This colonist is then removed from
play for the rest of the round.

The colonist pick phase creates a situation where players have asymmetrical information.
For example, the first player knows which colonist was removed at the start, but has
no information about the other players apart from knowing the four colonist he is passing
on. In contrast, the last player has relatively little information about the players
before them, but they know which colonist is removed from play after being left over.

\subsection{Proper Turns}

After each player has chosen a colonist, the players take their actual turns, in order
of first to last. Each player acts in all phases of their turn before passing the turn
to the next player.

Each turn consists of the following phases:
\begin{itemize}
    \item \emph{Draw Phase}. The player has two options in this phase.
        They may acquire two Omnium, which is the game's currency. The player's
        Omnium count persists between rounds. Omnium is used to build modules in
        the player's colony.
        The player may also opt to draw 2 modules from the deck. The player
        must then keep one of the modules, and place the other at the bottom of the deck.
        The drawing action is not available if the player's hand is full (five modules).
        If the player controls a~colonist with a~passive ability, this ability
        is automatically performed at the start of the draw phase.
        The player's colonist is also revealed to other players during this phase.
    \item \emph{Power Phase}. The player may choose to use their colonist's active
        ability if the colonist has one.
    \item \emph{Build Phase}. The player may choose to build one module from their hand.
        To build a module, the player must spend the Omnium amount required by the module's
        build cost. Building the module adds it to the player's colony and removes
        it from their hand.
\end{itemize}

\clearpage
\subsection{Colonists}
\label{gamedesign:colonists}

The following colonists and their respective abilities are available in \emph{Colonizers}:
\begin{itemize}
    \item Visionary
        \begin{itemize}
            \item Passive Ability: Draw a card if the player's hand is not full
                (maximum hand capacity is five).
            \item Active Ability: None
        \end{itemize}
    \item Ecologist
        \begin{itemize}
            \item Passive Ability: Gain 1 Omnium for each green module in his colony.
            \item Active Ability: None
        \end{itemize}
    \item Miner
        \begin{itemize}
            \item Passive Ability: Gain 1 Omnium for each blue module in his colony.
            \item Active Ability: None
        \end{itemize}
    \item General
        \begin{itemize}
            \item Passive Ability: Gain 1 Omnium for each red module in his colony.
            \item Active Ability: None
        \end{itemize}
    \item Opportunist
        \begin{itemize}
            \item Passive Ability: None
            \item Active Ability: Steal up to 2 Omnium from a chosen colonist.
                If no player controls the chosen colonist, this ability has no effect.
        \end{itemize}
    \item Spy
        \begin{itemize}
            \item Passive Ability: None
            \item Active Ability: Swap hands with a chosen colonist.
                If no player controls the chosen colonist, this ability has no effect.
        \end{itemize}
\end{itemize}

\clearpage
\subsection{Modules}
\label{de:modules}

At the start of the game, all modules start in a~deck in a~random order.
The order of modules in the deck is hidden. Whenever a~module is drawn by
a~player, it is taken from the top of the deck. Whenever a~module is discarded, it
is placed at the bottom of the deck. The only exception to this is the situation where
a player is discarding a card after drawing two in the draw phase. If the player's
colonist is the Visionary, it is possible for the player to overdraw (draw more modules
than the hand capacity would allow). In this situation, overdrawn modules are removed
from play entirely for the rest of the game.

There are 52 modules in the game. Modules have no special effects on the game board,
the only interaction the player has with them is building them.
The modules built in the player's colony also influence the passive ability
of the following colonists: Ecologist, Miner, General. The colored modules
are intentionally less efficient than modules without a color. This is because
they enable synergies with certain colonists. As a consequence, the game has a dynamic
where in the early game, it is often beneficial to build colored modules for synergy
in order to establish a good Omnium economy for future turns. As the game draws closer
to the end, it is often best to start disregarding synergy and simply build the most
efficient buildings points-wise.
\Cref{tabde:modules} is a table of all modules available in the game.
Note that the module names are a cosmetic feature which would be welcome in a physical
copy of the game, but due to screen space constraints on a computer screen,
module names were omitted from the game's user interface.

\begin{table}[h!]
    \centering
    \begin{tabular}{l@{\hspace{0.5cm}} c c c c}
    \textbf{Name} & \textbf{Build Cost} & \textbf{Value} & \textbf{Color} & \textbf{Quantity} \\
    \midrule
    Oxygen Generator            & 4  & 4   & Green   & 4 \\
    Water Reservoir             & 5  & 6   & Green   & 4 \\
    Hydroponics Facility        & 6  & 8   & Green   & 4 \\
    Eco-Dome                    & 8  & 11  & Green   & 1 \\
    Marketplace                 & 2  & 2   & Blue    & 4 \\
    Warehouse                   & 3  & 3   & Blue    & 4 \\
    Quarry                      & 5  & 6   & Blue    & 4 \\
    Omnium Purification Plant   & 8  & 10  & Blue    & 1 \\
    Garrison                    & 1  & 1   & Red     & 4 \\
    Barracks                    & 2  & 2   & Red     & 4 \\
    Military Academy            & 3  & 3   & Red     & 4 \\
    Planetary Defense System    & 6  & 7   & Red     & 1 \\
    Housing Unit                & 1  & 1   & None    & 4 \\
    Spaceport                   & 4  & 5   & None    & 4 \\
    Research Lab                & 6  & 8   & None    & 4 \\
    Mass Relay                  & 12 & 16  & None    & 1 \\
    \bottomrule
    \end{tabular}
    \caption{Available modules.}\label{tabde:modules}
\end{table}

\section{Branching Factor}
\label{sec:branching}

The branching factor of a~game is an important statistic which has heavy
influence on the performance of many game-playing algorithms \cite{Michalewicz04}.
Therefore it is useful to analyze the branching factor present in \emph{Colonizers}.

Firstly, we can analyze the branching factor present during the colonist pick phase.
There are six colonists in the game, and one of them is always randomly
removed. Then, players take turns picking from the remaining five colonists.
This means that the first player may choose from five colonists, and the last
player may choose from only two. We can view the distribution of colonists during the
pick phase as a permutation of colonists, therefore there are $6! = 720$
different outcomes for the colonist pick phase. An interesting case are permutations
where the first and last colonist (the colonists removed from play) are swapped.
Even though both end up out of play for the round, their presence during the pick
phase has an~observable effect on the players' information sets. Therefore we must
consider these permutations as distinct.

During the draw phase, a~player may either gain Omnium, or draw two cards and discard
one of them. This means that the draw phase for a single player has $3$ different outcomes.

In the power phase, branching factor is only relevant for colonists with active abilities,
namely Spy and Opportunist. Players controlling either of these colonists may choose
to either not use their power, or to target one of the five remaining colonists.
This gives us $6$ possible choices. For other colonists, the branching
factor is $1$. Since two out of the six colonists have an active ability,
we can say that the average branching factor for the power phase is
$\frac{16}{6} \approx 2.67$.

Lastly, we will consider the build phase. A~player may always choose to build nothing.
Since the hand size is limited to five, a~player may choose to perform up to six
different actions during the build phase.

Altogether, this gives us an approximate branching factor of $3 * \frac{16}{6} * 6 = 48$
per player turn taken. Since each player takes a turn during a round, we have
a branching factor of $48^{4} = 5 308 416$ per round. If we also consider the colonist pick phase,
we have an~approximate branching factor of $5 308 416 * 720 = 3 822 059 520$ for a~complete
round --- nearly four billion outcomes. For reference, game length usually ranges
between from about 15 to 40 rounds.
\footnote{We did not conduct exact measurements for game length, these values are simply
approximate observations.}.
