\chapter{AI Framework}

This chapter contains a high-level overview of the AI Framework in \emph{Colonizers}.
More concrete descriptions are available in \Cref{sec:userdocs} and \Cref{sec:devdocs}.

\section{Design}

Firstly, the choice of technologies used for creating \emph{Colonizers}
warrants some explanation. The architecture of the application comprises of three layers:
\begin{itemize}
    \item \emph{Game Engine}. The game's backend is implemented in C\# and targets
        .NET Core 3.1. The game logic itself is separated into a~library,
        and this library is then hosted behind a~web API with ASP.NET Core.
    \item \emph{UI Layer}. The UI for the application is a~single-page application
        (SPA) made with Angular.
    \item \emph{AI Scripts}. The AI scripts are implemented in Python 3.7. The project
        contains a base class \texttt{AIBase} for other AIs. This base class provides
        the AIs with the necessary API for communicating with the game engine.
\end{itemize}

The application is then hosted in Electron in order to run as a desktop application.
This is facilitated by the C\# library Electron.NET, which allows the hosting
of ASP.NET Core applications inside Electron. There are multiple reasons why
we chose Electron instead of a more traditional GUI like WPF or WinForms,
and instead of rendering graphics for the game ourselves:
\begin{itemize}
    \item Electron is cross-platform. The attentive reader has probably noticed that
        all three of the aforementioned components are made with cross-platform
        technologies, and this is very much by design. In principle, nothing
        prevents \emph{Colonizers} from being cross-platform, which is advantageous
        considering many AI researchers have Linux as their primary platform.
        In practice however, \emph{Colonizers} only supports Windows due to
        the way that communication between the game engine and AI scripts
        is implemented. This communication channel could be easily replaced
        with a cross-platform solution, in fact the classes which are responsible
        for communication could easily be swapped out. The only reason why
        this has not been done is simply prioritization --- the application has
        other features which had a higher priority.
    \item Electron provides easy ways of bundling and installing applications.
        It has a convenient installer which makes installing the application
        easy for end users.
    \item Electron is a stable and tested technology, which is used by
        many widely-used applications.
    \item A turn-based board game lends itself well to being drawn in a web page
        and being controlled by a SPA. If \emph{Colonizers} were an action game or
        a real-time strategy game, this solution would no longer be viable.
\end{itemize}

C\# was chosen for the implementation of the game engine because it is a~language
well-suited for writing backends. There are many solid libraries, and the language
is fast thanks to a~well-optimized JIT (Just In Time) compiler. It would have
been somewhat easier to implement the game's backend in Python, since the communication
between the game engine and the AI scripts would have been trivial. However, C\# was chosen
mostly because it is a statically-typed language, which offers protections against many
types of bugs and errors by catching them at compile-time. When writing game logic,
many errors are prevented simply by clever usage of types, as opposed to them
coming out during runtime with Python.

The choice of Python for AI scripts was an easy one, considering most machine learning
is done with Python today. It is also a~rather easy language to learn and understand.

Angular was chosen as the SPA framework because it is a~popular and robust solution.
There are multiple such frameworks, notably React and Vue, which would have also
made viable choices. However, Angular is simply the SPA framework the author
is the most familiar with.

Another design consideration is the design of the API used to communicate between
the game engine and AI scripts. The chosen API is subclassing --- a~new AI
must simply subclass a provided base class (\texttt{AIBase}), and implement
an abstract method. The AI script is then started in a~separate process
by the game engine, and the implemented method is called when the game engine
requires the AI to make a~move. The AI then chooses a~move, and returns an~identifier
corresponding to the chosen move. This API is simple and easy to understand,
as well as easy to implement.

\section{Interface}

The API for communication consists of the AI subclassing the \texttt{AIBase} class,
and implementing the \texttt{messageCallback(self, gameState)} abstract method. This method
is invoked by the game engine when it requires a~move to be chosen by the AI. The
\texttt{gameState} parameter contains a dictionary representing the current game state,
as well as a~list of available actions in this state.

The AI base class also contains various utilities for AI scripts to use. The most
important of these utilities is the \texttt{simulate(self, boardState, move)} method.
This method will simulate the given move via the game engine, and return the new game state.
This allows AIs to simulate playouts without having to implement them internally.
Another useful method is \texttt{determinize(self)}, which returns a~determinized
version of the current game state. This determinization is performed by the game
engine. The AI also does not need to track information sets, since the game engine
does this internally. This means that for example if the AI is playing second and
choosing a~colonist, the game engine will automatically track the information the AI
has about the previous player's colonist (one of two possible colonists).
Other utility methods included simplify working with the game state dictionary by
providing methods for commonly performed operations.

\clearpage
\Cref{framework:randomai} shows the implementation of an~AI for \emph{Colonizers}.
This AI simply chooses random moves.

\begin{figure}[h!]
\begin{code}[commandchars=\\\{\},codes={\catcode`\$=3\catcode`\^=7\catcode`\_=8}]
from AICore import *

import sys
from random import seed, randint

class RandomAI(AIBase):
    def \_\_init\_\_(self):
        super().\_\_init\_\_()

    def messageCallback(self, gameState):
        # important to return string, not number
        return str(self.pickRandomAction(gameState))

    def pickRandomAction(self, gameState):
        actionCount = len(gameState["Actions"])
        return randint(0, actionCount - 1)

if \_\_name\_\_ == "\_\_main\_\_":
    if len(sys.argv) != 2:
        # AI Script must have 1 argument - name of named pipe
        raise Exception('Invalid arguments')
    seed(42) # Seed AI for reproducibility
    ai = RandomAI()
    ai.run(sys.argv[1])
\end{code}
\caption{Random AI implementation.}\label{framework:randomai}
\end{figure}

As seen in \Cref{framework:randomai}, creating a new AI is very simple.
About half of the shown code is boilerplate code, and the actual AI
class is very simple.

Note that the game engine invokes the AI script via command line, which
means that the AI must contain the \texttt{\_\_main\_\_} boilerplate code.

\section{Adding new AIs}

There are two supported kinds of AI scripts: a~stand-alone Python file,
and a folder with potentially multiple Python files and a complicated subdirectory
structure. The second option exists to facilitate more complicated AIs, where
implementing the whole AI in a single Python file would not be reasonable.

The game looks for AI files in an~internal directory inside its installation folder.
While it is possible to add and remove scripts this way, the GUI provides a~way
to do this easily. In order for an~AI script to be recognized by the game, it must not
only be located in the internal directory, but it must also follow certain conventions.

For stand-alone Python files, the only convention is that the file must follow the naming
convention of \texttt{<Name>Intelligence.py}. An example of this
is \texttt{RandomIntelligence.py}.

For folder-based AIs, the folder itself must follow the naming convention of
\texttt{<Name>Intelligence}, for example \texttt{NestedIntelligence}. This
folder must also directly contain a~file named \texttt{main.py}, which will
be used as the AI's entrypoint.

The AI needs to reference the \texttt{AICore.py}
file in some way, since it contains the AI base class. This file is contained in
the aforementioned internal directory, which means that stand-alone AI scripts
placed in this directory can import the file without issues. The problem comes
with folder-based AIs, since Python unfortunately does not provide a~convenient way
to import files which are higher in the file hierarchy. Therefore when copying
AI folders with the GUI, the \texttt{AICore.py} file will automatically be copied
into the new AI's directory. This means that folder-based AIs for \emph{Colonizers}
cannot contain a \texttt{AICore.py} file in the top level of their folder,
since the file would be overwritten during copying. Alternatively, the folder
can be manually copied into the game files. In that situation however, the
\texttt{AICore.py} file must be copied manually as well.
