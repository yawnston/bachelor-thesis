\chapter{Used Algorithms}

\emph{Colonizers} has four different kinds of AI implemented out-of-the-box.
This chapter describes their implementations, and discusses the design decisions
taken when creating them.

\section{Random Decisions}

The random decision algorithm is rather primitive --- when it is presented with
a~choice of actions, it simply picks a random one. It is meant to be
the bottom baseline for other algorithms,
as well as being a proof-of-concept. \autoref{algo:random} shows the AI class which
implements this logic.

\begin{figure}[h!]
\begin{code}[commandchars=\\\{\},codes={\catcode`\$=3\catcode`\^=7\catcode`\_=8}]
class RandomAI(AIBase):
    def \_\_init\_\_(self):
        super().\_\_init\_\_()

    def messageCallback(self, gameState):
        # important to return string, not number
        return str(self.pickRandomAction(gameState))

    def pickRandomAction(self, gameState):
        actionCount = len(gameState["Actions"])
        return randint(0, actionCount - 1)
\end{code}
\caption{Random choice algorithm.}\label{algo:random}
\end{figure}

An interesting property of this AI is the fact that if four of them play
against each other, it is possible for the game to never end, since
the random decision making does not have to converge towards an~end state.
This situation is extremely unlikely however.

\section{Heuristics}

The heuristic AI is intended to be the real baseline for other implemented AI algorithms.
It comprises of a number of rules which determine the action to perform in a given
game state. If no rules are applicable to a given state, the AI simply
falls back to random choice. \autoref{algo:heur} shows high-level pseudocode for
the implemented heuristic AI.

\begin{figure}[h!]
\begin{code}[commandchars=\\\{\},codes={\catcode`\$=3\catcode`\^=7\catcode`\_=8}]
if game phase is "ColonistPick":
    if player has at least 3 modules of the same color in their colony:
        pick color synergy colonist if available
    if player has 0-1 modules in hand:
        pick Visionary if available
    pick randomly
else if game phase is "Draw":
    if player has 0 Omnium or at least 4 modules in hand:
        acquire Omnium
    if player has 0 modules in hand:
        draw modules
    pick randomly
else if game phase is "Discard":
    if any player has 7 or 8 modules in their colony:
        keep the highest value module the player can afford
    if the player has 5+ modules in their colony:
        keep the module with the highest difference of $value - cost$
        that the player can afford
    keep the module with the most color synergy with the player's
    colony if possible

    pick randomly
else if game phase is "Power":
    if player's colonist is Opportunist:
        choose the most valuable player to steal from,
        where value is calculated as
        $Omnium - number of possible colonists for that player + 1$
        randomly choose a colonist this player could have,
        then steal from this colonist
    if player's colonist is Spy:
        find player with more cards than the current player
        and with sufficient information about their colonist
        if this player exists:
            swap hands with them
        else:
            do nothing
    pick randomly
else if game phase is "Build":
    if any player has 7 or 8 modules in their colony:
        build the module with the highest possible value
    if the current player has 5+ modules:
        build the module with the highest difference of $value - cost$
    build the module with the most color synergies if possible

    build randomly
\end{code}
\caption{Heuristic algorithm pseudocode.}\label{algo:heur}
\end{figure}

\clearpage
\section{MaxN}
\label{sec:algomaxn}

\section{Information Set Monte Carlo Tree Search}
