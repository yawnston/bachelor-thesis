\chapter*{Introduction}
\addcontentsline{toc}{chapter}{Introduction}

\section*{Foreword}
\addcontentsline{toc}{section}{Foreword}

In game theory, perfect information two-player games are often
studied, and numerous algorithms have been designed with the purpose
of playing them. This includes games like Chess and Go, which have had
large breakthroughs in recent years \cite{Silver16}. However, real world
situations do not always have perfect information, or only two parties involved.
We could for example imagine multiple countries, which have only approximate
information about the armies of their opponents. In this scenario, it could be
useful to have tools to simulate potential enemy troop movements or placements.

Even though algorithms which are able to model imperfect information and multiple
players are often useful, they are not studied nearly as often. Designing such
an~algorithm is not easy, and there are many pitfalls which make conventional
game theory algorithms much less effecive at solving imperfect information and
multi-player problems. This thesis therefore aims to analyze the problems
of~implementing such algorithms, and to~implement some of them in~pursuit of~that goal.

Naturally, some frameworks do already exist for the implementation of such games.
However, at~the~time~of~writing, some~of~them only have AI (Artificial Intelligence)
support as~an~experimental and sparsely documented feature \cite{Boardgameio},
and others only focus on~specific fields of AI \cite{Openaigym}. This work aims to provide 
a~kind~of plug-and-play" experience, where AI developers have minimal barriers between
cloning a~git repository and having a~working AI.

\section*{Goals}
\addcontentsline{toc}{section}{Goals}

The main goal of this thesis is to create a multi-player board game with
imperfect information states. The game's name is \emph{Colonizers}. The game will primarily be designed with AI
in mind, and it will provide a reasonable interface for
the implementation of AI players.

Another goal is the~implementation of several AI players for said game. This will
allow us to not~only explore potential problems with implementing AIs for games
of this kind. We will also verify that the~API (Aplication Programming Interface)
provided by the game is sufficient for implementation of such AI players, and that
the~API is reasonably easy to use.

We then wish to compare the implemented AI algorithms in mutual play,
and identify the algorithms which perform the best. This will establish a benchmark
for future AI algorithms which can be developed for \emph{Colonizers}.
